 
%%-------------------------------------------------------------------- 
%%-------------------------------------------------------------------- 

\begin{center}
\par\textbf{\huge Modes de convergence}
\medskip
\par\textit{\Large Compléments à la fiche \no\thenumRecap}
\bigskip
\end{center}


Cette petite fiche récapitule le vocabulaire associé aux différentes
façon de converger pour une suite ou une série. On peut à chaque fois
se ramener à la convergence d'une suite \textbf{positive} vers $0$, ce qu'on
notera \fbox{$a_n\to 0$}.


\subsection*{Suite / série  numérique}

$(u_n)_n$ est une suite de $\K$, avec $\K=\R$ ou de $\K=\C$.

\smallskip

$\sum
u_n$ est une série numérique de terme général $u_n$. En notant
$S_n=\dsum_{k=0}^nu_n$,
$(S_n)_n$ est la suite numérique des sommes partielles de $\sum u_n$.

\smallskip

\begin{itemize}
  \setlength\itemsep{0.6em}
\item La suite $(u_n)_n$ \textbf{converge}: il existe $\ell\in\K$ tel que
  \fbox{$|u_n-\ell|\to 0$}. On dit que $(u_n)_n$ converge vers
  $\ell=\lim u_n$.
\item La série $\sum u_n$ \textbf{converge}: il existe $S\in\K$ tel que
  \fbox{$|S_n-S|\to 0$}. On dit que $\sum u_n$ converge et a pour
  somme $S=\dsum_{n=0}^{+\infty}u_n$.
\item La série $\sum u_n$ \textbf{converge absolument}: la série positive $\sum
  |u_n|$ converge.
\end{itemize}

\smallskip

{\bfseries Théorème:} La convergence absolue implique la convergence.


\subsection*{Suite / série vectorielle en dimension finie}


$E$ est un $\K$-espace vectoriel \textbf{de dimension finie}
(exemple: $E=\M_n(\K)$). $\|\cdot\|$ est une norme quelconque sur $E$
(elles sont toutes équivalentes).

\smallskip

$(u_n)_n$ est une suite de $E$.

\smallskip

$\sum u_n$ est une série vectorielle de
terme général $u_n$. En notant 
$S_n=\dsum_{k=0}^nu_n$,
$(S_n)_n$ est la suite vectorielle des sommes partielles de $\sum u_n$.

\smallskip

\begin{itemize}
  \setlength\itemsep{0.6em}
\item La suite $(u_n)_n$ \textbf{converge}: il existe $\ell\in E$ tel que
  \fbox{$\|u_n-\ell\|\to 0$}. On dit que $(u_n)_n$ converge vers
  $\ell=\lim u_n$ - cela ne dépend pas de la norme utilisée.
\item La série $\sum u_n$ \textbf{converge}: il existe $S\in E$ tel que
  \fbox{$\|S_n-S\|\to 0$}. On dit que $\sum u_n$ converge et a pour
  somme $S=\dsum_{n=0}^{+\infty}u_n$.
\item La série $\sum u_n$ \textbf{converge absolument}: la série positive $\sum
  \|u_n\|$ converge - cela ne dépend pas de la norme utilisée.
\end{itemize}

\smallskip

{\bfseries Théorème:} La convergence absolue implique la convergence.


\subsection*{Suite / série de fonctions réelles ou complexes}

$I$ est un intervalle de $\R$. Si $g:I\to \K$ est une fonction bornée,
on notera $\|g\|_{\infty}=\dsup_{x\in I}|g(x)|$. Pour $a,b\in I$, avec
$a<b$, on notera, lorsque c'est possible:
\[ \|g\|_{\infty}^{[a,b]}=\dsup_{x\in [a,b]}|g(x)|\qquad
\|g\|_1^{[a,b]}=\dint_a^b|f(t)|\d t\qquad
\|g\|_2^{[a,b]}=\sqrt{\dint_a^b|f(t)|^2\d t}\]
.

\smallskip

$(f_n)_n$ est une suite de fonctions définies sur $I$ et à valeurs
numériques: $f_n:I\to\K$.

\smallskip

$\sum f_n$ est 
une série de fonctions de terme général $f_n$. En notant 
$S_n=\dsum_{k=0}^nf_n$,
$(S_n)_n$ est la suite de fonctions des sommes partielles de $\sum
f_n$.

\smallskip

\begin{itemize}
  \setlength\itemsep{0.6em}
\item La suite $(f_n)_n$ \textbf{converge simplement sur $I$}: il existe
  une fonction $f:I\to\K$ telle que pour tout $x\in I$,
  \fbox{$|f_n(x)-f(x)|\to 0$}. On dit que $(f_n)_n$ converge
  simplement vers $f$ sur $I$, $f$ est appelée limite (simple) de la
  suite $(f_n)_n$.
\item La série $\sum f_n$ \textbf{converge simplement sur $I$}: il existe
  une fonction $S:I\to\K$ telle que pour tout $x\in I$,
  \fbox{$|S_n(x)-S(x)|\to 0$}. $S$ est appelée somme (simple) de la
  série $\sum f_n$.
\item La suite $(f_n)_n$ \textbf{converge uniformément sur $I$}: il existe
  une fonction $f:I\to\K$ telle que
  \fbox{$\|f_n(x)-f(x)\|_{\infty}\to 0$}. On dit que $(f_n)_n$ converge
  uniformément vers $f$ sur $I$, $f$ est appelée limite (uniforme) de la
  suite $(f_n)_n$.

\item La série $\sum f_n$ \textbf{converge uniformément sur $I$}: il existe
  une fonction $S:I\to\K$ telle que
  \fbox{$\|S_n(x)-S(x)\|_{\infty}\to 0$}. $S$ est appelée somme (uniforme) de la
  série $\sum f_n$.

\item La série $\sum f_n$ \textbf{converge normalement sur $I$}: La
  série numérique positive $\sum \|f_n\|_{\infty}$ est convergente.

\end{itemize}

\smallskip


{\bfseries Remarque:} On a la convergence normale ou uniforme \textbf{sur
  tout segment} en remplaçant $\|\cdot\|_{\infty}$ ci-dessus par
$\|\cdot|_{\infty}^{[a,b]}$ pour tout $a,b\in I$ tels que $a<b$.


\smallskip




{\bfseries Théorèmes:}
\begin{itemize}
  \setlength\itemsep{0.6em}
\item De façon générale la convergence normale ou uniforme sur $I$
  implique la convergence normale ou uniforme sur tout segment.
\item $(f_n)_n$ converge uniformément sur tout segment de $I$ vers $f$ $\implique$
  $(f_n)_n$ converge simplement sur $I$ vers $f$.
\item $\sum f_n$ converge normalement sur tout segment de $I$
  $\implique$ $\sum f_n(x)$ converge absolument pour tout $x\in I$
  $\implique$ $\sum f_n$ converge simplement sur $I$.
\item $\sum f_n$ converge normalement sur $I$ (\emph{resp.} sur tout
  segment de $I$) $\implique$ $\sum f_n$ converge uniformément sur $I$ (\emph{resp.} sur tout
  segment de $I$). {\itshape (la preuve consiste à introduire d'abord
  la somme $S:I\to\K$ grâce la convergence simple, ce qui
  permet ensuite de considérer les restes $R_n=S-S_n$ pour montrer la
  convergence uniforme de la suite $(R_n)_n$ vers la fonction nulle)}
\end{itemize}

\smallskip

D'autres modes de convergences (encore !): si les $f_n$ sont continues
sur $[a,b]$:
\begin{itemize}
  \setlength\itemsep{0.6em}
\item La suite $(f_n)_n$ \textbf{converge en moyenne} sur $[a,b]$:
  il existe $f:[a,b]\to \K$ telle que \fbox{$\|f_n-f\|_1\to
    0$}. Autrement dit $\dint_a^b|f_n(t)-f(t)|\d t\to 0$.
\item La suite $(f_n)_n$ \textbf{converge en moyenne quadratique} sur $[a,b]$:
  il existe $f:[a,b]\to \K$ telle que \fbox{$\|f_n-f\|_2\to
    0$}. Autrement dit $\dint_a^b|f_n(t)-f(t)|^2\d t\to 0$.
\end{itemize}

Remarque: la convergence uniforme sur $[a,b]$ implique la convergence
en moyenne quadratique, qui implique à son tour la convergence en
moyenne. Les réciproques sont fausses. Cela résulte de la comparaison
des normes:
\[\|\cdot\|_1^{[a,b]}\quad\le\quad
  \sqrt{b-a}\|\cdot\|_2^{[a,b]}\quad\le\quad
  (b-a)\|\cdot\|_{\infty}^{[a,b]}\]
On n'a pas cependant des inégalités dans l'autre sens: aucune de ces
normes ne sont équivalentes entre elles. \textbf{Préciser le mode de
  convergence our la norme utilisée pour une suite ou série de
  fonctions est essentiel !}